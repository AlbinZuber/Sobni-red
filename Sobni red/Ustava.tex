\documentclass[a4paper]{article}
\usepackage[slovene]{babel}
\usepackage[utf8]{inputenc}
\usepackage[T1]{fontenc}
\usepackage[margin=2cm]{geometry}
\setlength{\parskip}{12pt}
\title{USTAVA SOB 203 \& 204}
\author{}
\makeindex
\begin{document}

\newgeometry{left=3cm,bottom=1cm}



\maketitle

\begin{center}
\Large\title{§1}

\large Vsi stanovalci sob 203 \& 204 se strinjajo, da bodo k vsem sporom, nesporazumom ali drugim nesoglasjem pristopili z iskrenostjo in voljansotjo, da se konflikt razreši hitro, učinkovito in v interesu vseh strank.

\Large\title{§2}

\large Stanovanje bo delovalo po modelu federativne republike, kjer imajo stanovalci zakonodajno, izvršno in pravosodno oblast le v tisti sobi, kjer živjo.

\large Skupni prostori stanovanja so v zakonodajalni, izvršni in pravosodni oblasti vseh stanovalcev.

\large Mejo med skupnimi prostori označuje podboj vrat, ki bo deloval kot meja med sobo in skupnimi prostori.


\Large\title{§3}

\large Za ustavni amandma t.j. spremembo ustave, je potrebna stroga večina glasov. Vsak stanovalec dobi en glas.

\large Vse osebe, katerim je določen prijateljski, romantični ali katerikoli drugi razmerni status, dobijo polovico glasu.

\Large\title{§4}

\large V primeru, da stanovalec meni, da se mu je zgodila krivica, ima pravico tožbe proti stranki, ki je to krivico domnevno zagrešila.

\large Pri vsaki tožbi je obtožena stranka domnevno nedolžna, dokler ji krivda ni dokazana.

\large Sodnik ne sme biti osebno udeležen v tožbo. Če v stanovanju ni osebe ki ni udeležana, ali ne more ostati nepristranska iz drugih razlogov, vsaka od strank predlaga nadomestnika, ki je neodvisen in nepristranski po mnenju obeh strank.

\large Če neodvisen in nepristranski namestnik ni najden, se iz kandidatov sestavi pravosodni svet, za katerega vsaka stranka predloži največ enega kandidata. Več pravil v členih 5 \& 6

\Large\title{§5}

\large Pravosodni svet sestavljajo $pravosodni$ $komisarji$, ki morajo pred objavo skupne odločbe poslušati vse stranke udeležene v tožbo.

\large V skupni odločba naj jasno in eksplicitno piše odločitev sveta, razlogi za to odločitev in navodila za popravilo zagrešene krivice.

\large Vsebina skupne odločbe mora biti dostopna vsem stanovalcem, ki to želijo. 

\Large\title{§6}

\large Za pravosodni svet vsaka stranka vključena v tožbo predlaga enega kandidata.

\large Vsaka druga stranka ima veto pravico nad največ enim kandidatom in pravico ugovora čaz ostale kandidate.

\large Kandidat mora ustrezno odgovoriti na ugovore, če tega ni sposoben, njegova kandidatura propade.

\large Ko kandidat prestane vse ugovore zoper sebe in je izbran za pravosodni svet, dobi status pravosodnega komisarja, pri čemer dobi tudi naziv $pravosodni$ $komisar$, okrajšano z $prk$.

\Large\title{§7}



\end{center}
\end{document}